% Section 3 - Software tools of the trade
% Roberto Masocco <roberto.masocco@uniroma2.it>
% June 7, 2023

% ### Software tools of the trade ###
\section{Software tools of the trade}
\graphicspath{{figs/section3/}}

% --- OpenCV ---
\begin{frame}{OpenCV}{The de-facto stadard library for computer vision}
	\begin{columns}
		\column{.5\textwidth}
		\only<1>{
			\hspace{-.1cm}\href{https://opencv.org/}{\color{blue}\underline{\textbf{OpenCV}}} is the largest open-source library for real-time \textbg{computer vision} and \textbg{image processing}. It has been developed initially by Intel, and evolved in a \textbg{cross-platform} library supporting \textbg{C++} and \textbg{Python}.
			\newline\newline
			It ships with hundreds of \textbg{APIs} and \textbg{algorithms} to perform a wide variety of image processing and computer vision tasks and computations.
		}
		\only<2>{
			\hspace{-.25cm}In Linux, it is available as \textbg{binary packages}, but the full range of features can be enabled only by \textbg{configuring a source build}.
			\newline\newline
			Its \textbg{elementary data type} is a \textbg{matrix}: \texttt{cv::Mat}.
			\newline\newline
			It is \textbg{heavily optimized} to run on:
			\begin{itemize}
				\item \textbg{parallel CPUs};
				\item \textbg{GPUs} (\emph{e.g.}, \texttt{cv::cuda}, \texttt{cv::ocl});
				\item \textbg{embedded devices} (\emph{e.g.}, Nvidia Jetson).
			\end{itemize}
		}

		\column{.5\textwidth}
		\begin{figure}
			\centering
			\includegraphics[width=.6\textwidth]{opencv}
			\caption{OpenCV logo.}
			\label{fig:opencv}
		\end{figure}
	\end{columns}
\end{frame}

% --- ROS tools for image processing ---
\begin{frame}{ROS tools for image processing}{A quick overview}
	\textbg{ROS 2} provides a wide range of \textbg{tools} to acquire, process, and transmit images. The most important ones are:
	\begin{itemize}
		\item \texttt{sensor\_msgs/Image}: the standard ROS \textbg{message type} to transmit \textbg{images} over the DDS layer;
		\item \texttt{sensor\_msgs/CameraInfo}: the standard ROS \textbg{message type} to transmit and parse \textbg{camera calibration parameters};
		\item \href{https://wiki.ros.org/image_transport}{\color{blue}\underline{\texttt{image\_transport}}}: a ROS package to \textbg{transmit images} over the DDS layer;
		\item \href{https://wiki.ros.org/camera_info_manager}{\color{blue}\underline{\texttt{camera\_info\_manager}}}: a library to parse, use, and store \textbg{camera calibration parameters} from configuration files and messages inside ROS nodes.
	\end{itemize}
\end{frame}
\begin{frame}[fragile]{ROS tools for image processing}{The Image message}
	\begin{columns}
		\column{.9\textwidth}
		\begin{lstlisting}[language=ros2msg, caption=Definition of the \texttt{sensor\_msgs/msg/Image} message.]
std_msgs/Header header

uint32 height
uint32 width

# This can be RGB, BGR, RGBA, BGRA, YUV, Bayer, etc.
string encoding

uint8 is_bigendian
uint32 step
uint8[] data\end{lstlisting}
	\end{columns}
\end{frame}
\begin{frame}{ROS tools for image processing}{Sending images over the DDS}
	Using images and video streams in general over the DDS layer is \textbg{not trivial}, since:
	\begin{enumerate}
		\item we would like some kind of \textbg{specialized transport strategy} to optimize throughput and latency;
		\item the DDS specification is \textbg{not optimized} to transmit \textbg{large} and \textbg{variable-size} data chunks over potentially \textbg{lossy} networks, \emph{e.g.}, WiFi.
	\end{enumerate}
	\texttt{image\_transport} is a ROS 2 library that provides \textbg{wrappers} for \textbg{topic publishers} and \textbg{subscribers}, allowing to transmit images using different \textbg{transports}:
	\begin{itemize}
		\item \texttt{raw}: the default transport, which uses the \texttt{sensor\_msgs/Image} message;
		\item \texttt{compressed}: uses the \texttt{sensor\_msgs/CompressedImage} message, automatically converting frames to \texttt{JPEG} or \texttt{PNG} format;
		\item whatever you want to implement as a \textbg{plugin}.
	\end{itemize}
\end{frame}
\begin{frame}{ROS tools for image processing}{A note on topics and their statistics}
	\begin{alertblock}{Beware!}
		\begin{enumerate}
			\item When measuring image topics publishing rates with \texttt{ros2 topic hz}, especially \textbr{over a lossy network}, you may get \textbr{wrong results}. This is because of:
			      \begin{itemize}
				      \item the \textbr{Python DDS implementation};
				      \item the \textbr{hz} Python tool buffering messages into a window-based filter.
			      \end{itemize}
			      To check the real sampling and receiving rates, \textbr{place publishers of ordinary messages} in your code on both ends, and inspect those instead!\\
			      Suggestion: \texttt{std\_msgs/Empty}.
      \item Always use a \textbr{best effort} reliability policy in the \textbr{QoS} policy.
		\end{enumerate}
	\end{alertblock}
\end{frame}
